\partchapter{The Truth}{IV}

\lettrine{T}{he} spiralling leaves of the gigantic dieffenbachia felt like forest loam beneath Harry’s shoes, not as unyielding as concrete, but supporting his weight. Harry kept a wary eye on the tendrils, but they remained passive.

When Harry reached the bottom of the leafy spiral staircase, the tendrils suddenly whipped out and grasped Harry’s arms and legs.

After a brief struggle, Harry allowed himself to go limp.

“Interesting,” said Professor Quirrell, as he floated down from above, not touching any of the plant’s leaves or tendrils. “I notice that you seem to have no trouble losing to a plant.”

Harry looked more closely at the Defence Professor, seeing him now without the lens of panic. Professor Quirrell was upright and moving, flying without apparent difficulty; the sense of doom about him was strong. But his eyes were still sunken in the skull, his arms thin and wasted. The sickness had \emph{not} been a bluff, and the obvious hypothesis was that the Defence Professor had recently killed another unicorn to temporarily regain some strength.

And the Defence Professor was also speaking like the mask of Professor Quirrell, not like Lord Voldemort, which might not be a bad thing from Harry’s perspective. Harry didn’t know why—unless it was that the Defence Professor still needed him for something—but it certainly seemed to be in Harry’s own interests to play along.

“You specifically let me walk into this trap, Professor,” Harry answered, just the way he’d have spoken to Professor Quirrell. \emph{Roles, masks, remind him of how it was between us…} “On my own, I’d have used my broomstick.”

“Perhaps. How would an ordinary first-year solve this challenge? If they had their wand, that is.” The plant was now reaching tendrils out toward Professor Quirrell, but Professor Quirrell was hovering just out of their reach.

Harry had now remembered Professor Sprout talking about a Devil’s Snare plant, which the Herbology textbook had said liked cool, dark places like caves—though how that could be true of a leafy plant was anyone’s guess. “At a guess, I’d say this is a Devil’s Snare plant and it might retreat from light or heat. So maybe a first-year could use Lumos? Today I’d use \emph{Inflammare}, but I didn’t learn that spell until May.”

A twirl of the Defence Professor’s wand, and a pattern of sprays of liquid shot out from it, striking the plant near the bases of its tendrils, hitting with a quiet splat and then a quiet hissing. All the tendrils touching Harry frantically shot back and began to beat at the growing wounds appearing on the plant’s skin, as if trying to remove the pain-stimulus; something about the plant gave the impression that it was screaming soundlessly.

Professor Quirrell finished drifting downward. “Now it is afraid of light, heat, acid, and me.”

Harry stepped off the final leaves onto the floor, after a careful glance at his robes and then the floor to make sure that none of the acid had splashed anywhere. Harry had begun to suspect that Professor Quirrell was trying to make some sort of point, but Harry did not know what that point might be. “I thought we were on a mission, Professor. I can’t stop you, but is it \emph{smart} to spend this much time on messing with me?”

“Oh, we have time,” said Professor Quirrell, sounding amused. “There would be a great uproar if we were discovered here, guarded by an Inferius. You did not act like you had heard of such an uproar at your Quidditch match, before you arrived in this time and spoke to Snape as you did.”

A slight chill came over Harry, as he comprehended this. Anything he did to beat Professor Quirrell would have to \emph{not} disrupt the school, or at least the Quidditch game, because it \emph{hadn’t} disrupted the Quidditch game. Even if enough forces could be called in to subdue Lord Voldemort, it might not be easy to do it without Professor McGonagall or Professor Flitwick or anyone else at the Quidditch game noticing…

Fighting a smart enemy was hard.

And even so…even so it seemed to Harry that if he stood in Professor Quirrell’s shoes, he would not be having leisurely conversations and playing mind games. Professor Quirrell was gaining \emph{something} by taking his time here. But what? Was there some other process that had to run to completion?

“By the by, have you betrayed me yet?” said Professor Quirrell.

“\parsel{Have not betrayed you yet,}” Harry hissed.

The Defence Professor gestured pointedly with the gun he was now holding in his left hand, and Harry walked ahead to the great wooden door at the end of the room, and opened it.

\later

The next chamber was smaller in diameter, with a higher ceiling. The light shining out of the arched alcoves was white, instead of blue.

Around them whizzed hundreds of winged keys, beating frantically through the air. After watching for a few seconds, it became clear that only a single key was the golden colour of a Snitch—though it was moving slower than a Snitch in a real Quidditch game.

On the other end of the room was a door containing a large, prominent keyhole.

Against the left wall leaned a broomstick, the school’s workhorse Cleansweep Seven.

“Professor,” Harry said, staring up at the clouds and flocks of whizzing keys, “you said you would answer my questions. What exactly is all this about? If you think you’ve secured a door so that it won’t open without a key, you keep the key in a safe place and only give a copy to authorized entrants. You don’t \emph{give the key wings} and then \emph{leave a broomstick propped against the wall}. So what the heck are we doing in here and what is going on? It’s an obvious guess that the magic mirror is the only real factor guarding the Stone, but why the rest of this—and why encourage first-years to come here?”

“I am truly not sure,” said the Defence Professor. He had entered the room and taken up station well to Harry’s right, maintaining the distance between them. “But I shall answer, as I said I would. Dumbledore’s way is to do a dozen things which seem mad, and then only eight of them, or perhaps nine, conceal an inner meaning. My guess is that Dumbledore intends to make it seem like I am invited to send a student as my proxy. Precisely so that Lord Voldemort, as Dumbledore conceives of him, is less tempted to think himself clever by doing so. Imagine Dumbledore first considering the issue of how to ward the Stone. Imagine Dumbledore considering whether to set true dangers to guard the Mirror. Imagine him imagining some young student blundering through those dangers at my behest. I think that is what Dumbledore is trying to avoid, by making it seem as though that strategy is invited, and so not cunning. Unless, of course, I have misunderstood what Dumbledore thinks Lord Voldemort will think.” Professor Quirrell grinned, and it looked just as natural, on him, as any grin he’d shown Harry before. “Plotting does not come naturally to Dumbledore, but he tries because he must. To that task Dumbledore brings intelligence, dedication, the ability to learn from his mistakes, and an utter lack of native talent. He is marvellously hard to predict for that reason alone.”

Harry turned away, looking at the door on the opposite side of the room. \emph{It wasn’t a game to him, Professor.} “My guess is that the intended solution for first-years is to ignore the broomstick and use \emph{Wingardium Leviosa} to grab the key, since this isn’t a Quidditch game and there are no rules forbidding that. So what absurdly overpowered spell are you going to unleash on this one, then?”

There was a brief silence but for the whizzing of keys.

Harry took several steps away from Professor Quirrell. “I probably shouldn’t have said that, should I.”

“Oh, no,” Professor Quirrell said. “I think that is a quite reasonable thing to say to the most powerful Dark Wizard in the world when he is standing not a dozen paces from you.”

Professor Quirrell put his wand back into the sleeve of his other hand, the hand that sometimes held the gun.

Then the Defence Professor reached into his mouth and took out what appeared to be a tooth. He tossed the false tooth high in the air, and when it came down, it had transformed into a wand that sparked a strange sense of recognition in Harry’s mind, as though some part of him recognized that wand as being…part of him….

\emph{Thirteen and a half inches, yew, with a core of phœnix feather.} Harry had memorized the information when the wand-maker Olli-something had given it, because it had seemed like it might be Plot-Relevant. The event, and the thinking that had underlain it, both felt a lifetime distant.

The Defence Professor raised that wand, and traced in the air a flaming rune that was all jagged edges and malevolence; Harry took another instinctive step back. Then Professor Quirrell spoke. “Az-reth. Az-reth. Az-reth.”

The flaming rune began pouring out fire that was…\emph{twisted}, as though the jagged edges of the rune had become the nature of the fire itself. The fire was blazing crimson, shaded further red than blood, glowing as searingly intense as an arc-welder. That brilliance in that shade seemed \emph{wrong} in its own right, like nothing shaded so far red should give off that much light; and the searing crimson was shot through with veins of black that seemed to suck the light from the fire. Within the blackened fire, outlined in the interplay of crimson and darkness, animal shapes twisted wildly from one predator to another, cobra to hyena to scorpion.

“Az-reth. Az-reth. Az-reth.” When Professor Quirrell had repeated the word six times, a small cauldron-full of black–crimson fire had poured out.

The cursed fire slowed in its changes as Professor Quirrell locked eyes upon it, taking on a single form, the form of a blackened blood-burning phœnix.

And something told Harry with a terrible certainty that if that black burning phœnix met Fawkes, the true phœnix would die and never be reborn.

Professor Quirrell made a single gesture with his wand, and the blackened fire went soaring across the room. It met the door and its keyhole, and with a single sweep of crimson-burning wings, most of the door and part of the archway was consumed. Then the tainted crimson blaze swept on.

Harry had only a glance through the hole to see huge statues just beginning to raise swords and clubs, when the blackened fire came among them, and they cracked and burned.

When it ended, the blackened-fire phœnix swept back in through the hole, and hovered above Professor Quirrell’s left shoulder, the sun-intense crimson claws staying an inch from his robes.

“Go on ahead,” said Professor Quirrell. “It’s safe now.”

Harry walked forward, needing to invoke his dark side’s cognitive patterns in order to maintain calm enough to do it. Harry stepped over the glowing edges of the remaining part of the door, and gazed at a chessboard of ruined huge chess pieces. The alternating tiles of black and white marble on the floor started five metres after the ruined doorway, and extended from wall to wall, but stopped five metres short of the next door on the opposite side of the room. The ceiling was significantly higher than any of the statues should have been able to reach.

“I would guess,” Harry said, and his dark side’s cognitive patterns kept his voice calm, “that the intended solution is to fly over the statues using the broomstick from the previous room, since it wasn’t actually needed to get the key?”

From behind, Professor Quirrell laughed, and it was Lord Voldemort’s laugh. “Proceed,” said a voice grown colder and higher. “Go to the next room. I wish to see what you will make of what is there.”

\emph{Arranged by Dumbledore for first-years,} Harry reminded himself, \emph{it \emph{will} be safe,} and he walked across the ruined chessboard, laid his hand upon that door’s handle, and pushed it inward.

\later

Half a second later, Harry slammed the door and leapt back.

It took Harry several seconds to master his breathing, and master himself. From behind the door came continued loud bellows, and great slams as of a rock club pounding the floor.

“I suppose,” Harry said in a voice grown cold as well, “that since Dumbledore would hardly put a real mountain troll in there, the next challenge is an illusion of my worst memories. Like a Dementor, with the memory projected into the outside world. Very amusing, Professor.”

Professor Quirrell advanced himself toward the door, and Harry stepped well aside. Besides the sense of doom that was now strong about the Professor, Harry’s dark side or just plain instinct was advising him not to get anywhere near that black-crimson fire hovering above Professor Quirrell’s shoulder.

Professor Quirrell swung open the door, and looked in. “Hm,” Professor Quirrell said. “Just the troll, as you say. Ah, well. I had hoped to learn something about you more interesting than that. What lies within is a Kokorhekkus, also known as the common boggart.”

“A boggart? What does that—no, I suppose I know what it does.”

“A boggart,” Professor Quirrell said, and now his voice was again that of a Hogwarts Professor lecturing, “gravitates to dark enclosures that are rarely opened, such as a neglected cupboard in the attic. It seeks to be left alone, and it will manifest in whatever form it thinks will scare you away.”

“Scare me away?” Harry said. “I \emph{killed} the troll.”

“You leapt backward out of the room without thinking. A boggart seeks out the instinctive flinch, not the reasoned threat. Else it would have selected something more believable. In any case, the standard counter-Charm for a boggart is, of course, Fiendfyre.” Professor Quirrell gestured, and the blackened fire leapt off his shoulder and poured through the doorway.

From within the room there was a single squeak, and then nothing.

They advanced into the boggart’s former room, Professor Quirrell going first this time. With the seeming mountain troll gone, the room was just another huge chamber lit by sconces of cold blue light.

Professor Quirrell’s gaze seemed distant, thoughtful. He crossed the room without waiting for Harry, and swung open the door on the opposite wall of his own accord.

Harry followed after, and not closely.

\later

The next chamber contained a cauldron, a rack of bottled ingredients, chopping boards, stirring sticks, and the other apparatus of Potions. The light coming from the arched alcoves was white instead of blue, presumably because colour vision was important to Potions-brewing. Professor Quirrell was already standing next to the brewing apparatus, scrutinizing a long parchment he had picked up. The door to the next chamber was guarded by a curtain of purple fire that would have looked a lot more threatening, if it hadn’t seemed pale and weak by comparison to the blackened flame hovering over Professor Quirrell’s shoulder.

Harry’s suspension of disbelief had already checked out on holiday at this point, so he didn’t say anything about how real-world security systems had the goal of \emph{distinguishing} authorized from unauthorized personnel, which meant issuing challenges that behaved \emph{differently} around people who were or weren’t supposed to be there. For example, a \emph{good} security challenge would be testing whether the entrant knew a lock combination that only authorized people had been told, and a \emph{bad} security challenge would be testing whether the entrant could brew a potion according to written instructions that had been helpfully included.

Professor Quirrell tossed the parchment toward Harry, and it fluttered to the ground between them. “What do you make of this?” said Professor Quirrell, who then stepped back so that Harry could come forward and pick up the parchment.

“Nope,” Harry said after skimming the parchment. “Testing whether the entrant can solve a ridiculously straightforward logic puzzle about the order of the ingredients is still not a challenge that behaves differently for authorized and unauthorized personnel. It doesn’t matter if you use a more interesting logic puzzle about three idols or a line of people wearing coloured hats, you’re still completely missing the point.”

“Look at the other side,” said Professor Quirrell.

Harry turned over the two-foot parchment.

On the other side, written in tiny letters, was the \emph{longest} list of brewing instructions Harry had ever seen. “What on Earth—”

“A \emph{potion of effulgence}, to quench the purple fire,” Professor Quirrell said. “It is made by adding the same ingredients, over and over again, in slightly different ways. Imagine some eager young group of first-years, passing all the other chambers, thinking they are just about to reach the magic mirror, and then encountering this task. This room is the handiwork of the Potions Master indeed.”

Harry glanced pointedly at the blackfire shape on Professor Quirrell’s shoulder. “Fire can’t beat fire?”

“It can,” said Professor Quirrell. “I am not sure it should. Suppose this room is trapped?”

Harry did \emph{not} want to be stuck brewing this potion for laughs, or for whatever other reason Professor Quirrell was taking them through these chambers so slowly. The potions recipe had \emph{thirty-five} separate occasions for adding bellflowers, fourteen times to add ‘a lock of bright hair’…“Maybe the potion gives off a lethal gas that is fatal to adult wizards but not children. Or any of a hundred other deadly tricks, if we’re suddenly being serious. Are we being serious?”

“This room is the handiwork of Severus Snape,” Professor Quirrell said, once more looking thoughtful. “Snape is not a bystander in this game, not quite. He lacks Dumbledore’s intelligence, but possesses the killing intent that Dumbledore never had.”

“Well, whatever’s going on here, it doesn’t actually keep out children,” Harry observed. “Lots of first-years made it through. And if you can somehow keep out everyone \emph{except} children, then that, from Dumbledore’s perspective, forces Lord Voldemort to possess a child to enter. I don’t see the point, given their goals.”

“Indeed,” Professor Quirrell said, rubbing the bridge of his nose. “But see, boy, this room lacks the triggers and tripsigns that are upon the others. There are no subtle wards to be defeated. It is as if I am \emph{invited} to bypass the Potion and simply enter—but Snape knows that Lord Voldemort will perceive this. If in fact there was a trap laid for anyone who did not brew the potion, then it would be wiser to lay wards, and give no sign that this room was different from the others.”

Harry listened, frowning in concentration. “So…the only point of leaving off the detection webs is to make you \emph{not} bulldoze this room.”

“I expect Snape expects me to deduce that as well,” the Defence Professor said. “And past that point I cannot predict at what level he thinks I will play. I am patient, and I have given myself plenty of time for this endeavour. But Snape does not know me, he only knows Lord Voldemort. He has sometimes seen Lord Voldemort shriek in frustration, and act on impulses that appear counterproductive. Consider this matter from Snape’s perspective: it is the Potions Master of Hogwarts telling Lord Voldemort to be patient and follow instructions if he wants to enter, as though Lord Voldemort were a mere schoolboy. I would find it easy to comply, smiling the while, and take my vengeance later. But Snape does not know that Lord Voldemort finds it easy to think this way.” Professor Quirrell looked at Harry. “Boy, you saw me floating in the air by the Devil’s Snare, did you not?”

Harry nodded. Then he noticed his confusion. “My Charms textbook says that it’s impossible for wizards to levitate themselves.”

“Yes,” said Professor Quirrell, “that is what it says in your Charms textbook. No wizard may levitate themselves, or any object supporting their own weight; it is like trying to lift yourself up by your own bootstraps. Yet Lord Voldemort alone can fly—how? Answer as quickly as you can.”

If the question was answerable by a first-year student—“You had someone else cast broomstick enchantments on your underwear, then you Obliviated them.”

“Not quite,” said Professor Quirrell. “The broomstick enchantments require a long narrow shape, which must be solid. Cloth will not do.”

Harry’s eyebrows furrowed. “How long does the shape have to be? Can you attach some short broomstick rods to a fabric harness, and fly using those?”

“Indeed, at first I strapped enchanted rods to my arms and legs, but that was only to teach myself a new mode of flight.” Professor Quirrell drew back the sleeve of his robes, revealing the bare arm. “As you can see, I have nothing up my sleeve right now.”

Harry absorbed this further constraint. “You had someone cast broomstick enchantments on your \emph{bones}?”

Professor Quirrell sighed. “And that was one of Voldemort’s most feared feats, or so I am told. After all these years, and some amount of reluctant Legilimency, I still do not truly comprehend what is \emph{wrong} with ordinary people…But you are not one of them. It is time for you to begin contributing to this expedition. You have known Severus Snape more recently than I\@. Tell me your own analysis of this room.”

Harry hesitated, trying to look thoughtful.

“I will mention,” said Professor Quirrell, as the blackened-fire-phœnix on his shoulder seemed to extend its head and glare at Harry, “that if you knowingly allow me to fail, I will call it betrayal. I remind you that the Stone is key to Miss~Granger’s resurrection, and that I hold hostage the lives of hundreds of students.”

“I remember,” Harry said, and on the heels of this Harry’s wonderful inventive brain came up with a thought.

Harry wasn’t sure if he should say it.

The silence stretched.

“Have you thought of anything yet?” said Professor Quirrell. “Answer in Parseltongue.”

No, this was \emph{not} going to be easy, not against a smart opponent who could force you to tell the literal truth at any time. “Severus, at least the modern-day Severus, respects your intelligence a great deal,” Harry said instead. “I think…I think he might \emph{expect} Voldemort to believe that Severus wouldn’t believe that Voldemort could pass his test of patience, but Severus \emph{would} expect Voldemort to pass it.”

Professor Quirrell nodded. “That is a plausible theory. Do you believe it yourself? Answer in Parseltongue.”

“\parsel{Yes,}” Harry hissed. It might not be safe to withhold information, not even thoughts and ideas…“Therefore, the point of this room is to delay Lord Voldemort for an hour. And if I wanted to kill you, believing what Dumbledore believes, the obvious thing to try would be a Dementor’s Kiss. I mean, they think you’re a disembodied soul—are you, by the way?”

Professor Quirrell was still. “Dumbledore would not think of that method,” the Defence Professor said after a time. “But Severus might.” Professor Quirrell began to tap a finger against his cheek, his gaze distant. “You have power over Dementors, boy, can you tell me if there are any nearby?”

Harry closed his eyes. If there were voids in the world, he could not feel them. “None that I can sense.”

“Answer in Parseltongue.”

“\parsel{Do not sense life-eaters.}”

“But you were being honest with me when you suggested the possibility? You intended no clever trickery?”

“\parsel{Was honest. Not trick.}”

“Perhaps there is some means by which Dementors might be concealed, being told to leap out and eat a possessing soul if they see one…” Professor Quirrell was still tapping his cheek. “It is not impossible that I would qualify. Or it can be told to eat anyone who passes through this room too quickly, or anyone who is not a child. Bearing in mind that I hold Hermione and hundreds of other students hostage over you, would you use your power over Dementors to defend me, if a Dementor unmasked itself? Answer in Parseltongue.”

“\parsel{Don’t know,}” Harry hissed.

“\parsel{Life-eaters cannot destroy me, I think,}” hissed Professor Quirrell. “\parsel{And I will simply abandon this body if they approach too close. Shall return swiftly this time, and then there will be no stopping me. Will torture your parents for years, to punish you for baulking me. Hundreds of hostage students die, including those you call friends. Now I ask again. Will you use power over life-eaters to protect me, if life-eaters come?}”

“\parsel{Yes,}” Harry whispered. The sadness and horror that Harry had pushed down flared up again, and his dark side had no stored patterns for handling the emotions. \parsel{Why, Professor Quirrell, why are you like this…}

Professor Quirrell smiled. “That reminds me. Have you betrayed me yet?”

“\parsel{Have not betrayed you yet.}”

Professor Quirrell went over to the Potions equipment, and began chopping a root one-handed, the knife moving almost invisibly fast and with no apparent effort. The Fiendfyre phœnix drifted over to the opposite corner of the room and waited there. “All matters considered in their uncertainty, it seems wiser to expend the time to pass this room as a first-year would,” said the Defence Professor. “We may as well talk while we are waiting. You had questions, boy? I said that I would answer them, so ask.”

%  LocalWords:  Cleansweep Olli Az reth Kokorhekkus blackfire
