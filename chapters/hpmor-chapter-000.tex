\chapter*{Preface}
% from https://www.hpmor.com/chapter/1
% This is not a strict single-point-of-departure fic—there exists a primary point of departure, at some point in the past, but also other alterations. The best term I’ve heard for this fic is “parallel universe”.

Este texto contiene muchas pistas: algunas obvias, otras no tan obvias, detalles muy escondidos que me sorprendió ver que algunos lectores correctamente dedujeron, y una gran cantidad de pruebas que he ido dejando a plena vista. Esta es una historia racionalista; sus misterios pueden ser resueltos, y están diseñados para que se resuelvan.

El ritmo de la historia es el de una serie de ficción, es decir, el de una serie de televisión con un numero determinado de temporadas, cuyos episodios han sido individualmente escritos pero con una trama global que los abarca y que construye hacia una conclusión final.

% The story has been corrected to British English up to Ch. 17, and further Britpicking is currently in progress (see the /HPMOR subreddit).

All science mentioned is real science. But please keep in mind that, beyond the realm of science, the views of the characters may not be those of the author. Not everything the protagonist does is a lesson in wisdom, and advice offered by darker characters may be untrustworthy or dangerously double-edged.

Toda ciencia mencionada es ciencia real. Sin embargo, hay que recordar, fuera del reino de la ciencia, la visión de los personajes puede no ser la misma que la del autor. No todo lo que haga el protagonista es una lección en sabiduría, y todo consejo proporcionado por un personaje oscuro puede ser malintencionado o un peligroso arma de doble filo.

\chapter*{Author’s introduction}
% from https://www.hpmor.com/chapter/22

\section*{Something, somewhere, somewhen, must have happened differently…}

\begin{itemize}
	\item \textsc{Petunia Evans} married Michael Verres, a Professor of Biochemistry at Oxford.
	\item \textsc{Harry James Potter-Evans-Verres} grew up in a house filled to the brim with books. He once bit a math teacher who didn’t know what a logarithm was. He’s read \emph{Gödel, Escher, Bach} and \emph{Judgment Under Uncertainty: Heuristics and Biases} and volume one of \emph{The Feynman Lectures on Physics}. And despite what everyone who’s met him seems to fear, he doesn’t want to become the next Dark Lord. He was raised better than that. He wants to discover the laws of magic and become a god.
	\item \textsc{Hermione Granger} is doing better than him in every class except broomstick riding.
	\item \textsc{Draco Malfoy} is exactly what you would expect an eleven-year-old boy to be like if Darth Vader were his doting father.
	\item \textsc{Professor Quirrell} is living his lifelong dream of teaching Defense Against the Dark Arts, or as he prefers to call his class, Battle Magic. His students are all wondering what’s going to go wrong with the Defense Professor this time.
	\item \textsc{Dumbledore} is either insane, or playing some vastly deeper game which involved setting fire to a chicken.
	\item \textsc{Minerva Mcgonagall} needs to go off somewhere private and scream for a while.
\end{itemize}

% \begin{center}
% Presenting:
%
% \textsc{Harry Potter and the Methods of Rationality}
%
% You ain't guessin' where this one's going.
% \end{center}

\section*{Some notes}
The opinions of characters in this story are not necessarily those of the author. What warm!Harry thinks is \emph{often} meant as a good pattern to follow, especially if Harry thinks about how he can cite scientific studies to back up a particular principle. But not everything Harry does or thinks is a good idea. That wouldn’t work as a story. And the less warm characters may sometimes have valuable lessons to offer, but those lessons may also be dangerously double-edged.

If you haven’t visited \url{https://hpmor.com}, don’t forget to do that at some point; otherwise you'll miss out on the fan art, how to learn everything Harry knows, and more.

If you don’t just enjoy this fic, but learn something from it, then please consider blogging it or tweeting it. A work like this only does as much good as there are people who read it.

%  LocalWords:
