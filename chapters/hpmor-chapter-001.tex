\chapter{Un día de muy baja probabilidad}

\begin{chapterOpeningQuote}
	\noindent
	Bajo la luz de la luna brilla un diminuto fragmento de plata, una fracción de línea...

	\vspace*{2ex}
	(túnicas negras que caen)

	\vspace*{2ex}
	…sangre que se derrama en litros, y alguien que grita una palabra.
\end{chapterOpeningQuote}

\lettrine{C}{ada} centímetro de pared está cubierto por una estantería. Cada estantería tiene seis baldas que llegan casi hasta el techo. Algunas están apiladas hasta los topes con pesados libros: de ciencia, matemáticas, historia, y muchas otras cosas. Otras tienen dos capas de libros de ciencia ficción, con la capa posterior de libros apoyada en viejas cajas de pañuelos de papel o trozos de madera, de manera que se pueda ver la parte posterior de los libros tras aquellos que están al frente. Y sin embargo, no es suficiente. Los libros rebosan hacia las mesas y los sofás y se acumulan en montoncitos bajo las ventanas.
\authorsnotetext{Hago esto en mi propia casa.}

Este es el salón de la casa que habitan el eminente profesor Michael Verres-Evans, su mujer, la señora Petunia Evans-Verres, y su hijo adoptivo, Harry James Potter-Evans-Verres.

Hay una carta sobre la mesa del salón, junto a un sobre sin sellar hecho de amarillento pergamino, dirigida al \emph{Sr.~H.~Potter} en tinta verde esmeralda.

El profesor y su mujer están discutiendo, pero sin gritar. El profesor considera incivilizado gritar.

—Tienes que estar bromeando —dijo Michael a Petunia. Su tono reflejaba temor. Temor de que le estuviesen hablando en serio.

—Mi hermana era una bruja —repitió Petunia, asustada. Aún así, mantenía su postura—. Su marido era un mago.

—¡Es absurdo! —dijo Michael cortante—. Vinieron a nuestra boda, nos visitaron en Navidad—

—Les dije que no debías saberlo —susurró Petunia—. Pero es la verdad. Yo misma he visto cosas que—

El profesor puso los ojos en blanco.

—Cielo, soy consciente de que no estás familiarizada con la literatura escéptica. Uno no se da cuenta de lo fácil que es para un mago entrenado hacer creer lo que, a primera vista, parece imposible. ¿Te acuerdas de cuando enseñé a Harry a doblar cucharas? Si lo dices porque parecía que siempre podían leer tus pensamientos, a eso se le denomina lectura en frío—

—No era doblar cucharas.

—¿Y qué era entonces?

Petunia se mordió el labio.

—No puedo decírtelo así sin más. Pensarás que soy\dots —Tragó saliva—. Escucha, Michael. Yo... no siempre he sido así —dijo gesticulando hacia sí misma, como señalando su esbelta figura—. Lily me \emph{hizo} así. Pero porque... pero porque le \emph{supliqué}. Le supliqué durante años. Lily \emph{siempre} había sido más hermosa que yo, y por eso... y por eso era mezquina con ella, y así fue como consiguió su \emph{magia}. ¿Te imaginas cómo me pude sentir? Y le \emph{supliqué} que usase un poco de esa magia conmigo para que yo también pudiese ser hermosa, para que incluso careciendo de su magia, al menos pudiese ser bella.

Las lágrimas se agolpaban en los ojos de Petunia.

—Y Lily siempre se negaba, inventándose las más ridículas excusas, como que el mundo terminaría si ella era amable con su hermana, o que un centauro le había dicho que no... cosas de lo más ridículas, y yo la odiaba por ello. Y cuando me acababa de graduar de la universidad, salía por entonces con un chico, Vernon Dursley, estaba gordo y y era el único chico que me dirigía la palabra. Y me dijo que quería tener hijos, y que su primer hijo se llamaría Dudley. Y entonces pensé, \emph{¿qué clase de padre llamaría a su hijo Dudley Dursley?} Fue como si pudiese ver mi futuro, y no podía soportarlo. Entonces le escribí a mi hermana y le dije que si no me ayudaba, entonces yo me... —Petunia se detuvo.

—Es igual —dijo Petunia con una vocecita—, accedió. Me dijo que era peligroso, y le respondí que ya no me importaba, y entonces me bebí una poción y estuve enferma durante semanas, pero cuando mejoré se me limpió la piel y finalmente gané algo de peso y... era hermosa, la gente era \emph{amable} conmigo —se le quebró la voz—, y después de aquello no pude odiar más a mi hermana, sobre todo tras descubrir el trágico final al que su magia le había llevado...

—Cielo —dijo Michael con delicadeza—, enfermaste, ganaste algo de peso mientras reposabas, y tu piel se mejoró por sí sola. O quizá enfermar te hizo cambiar tu dieta y—

—Era una bruja —repitió Petunia—. Lo vi.

—Petunia —dijo Michael. El fastidio crecía en su voz—. Tú \emph{sabes} que eso no puede ser cierto. ¿Hace falta de verdad que te explique por qué?

Petunia se retorcía las manos. Parecía al borde de las lágrimas.

—Mi amor, sé que no puedo ganar discusiones contigo, pero por favor, tienes que creer en mí esta—

—\emph{¡Papá! ¡Mamá!}

Ambos se detuvieron y miraron a Harry como si hubiesen olvidado que había una tercera persona en la habitación. Harry respiró hondo.

—Mamá, tus padres \emph{no} tenían magia, ¿no?

—No —dijo Petunia, perpleja.

—Por lo tanto, nadie en tu familia creía en la magia cuando Lily recibió su carta. ¿Como se convencieron \emph{ellos}?

—Ah... —dijo Petunia—. Ese día no mandaron tan solo una carta. También vino un profesor de Hogwarts. Él... —Los ojos de Petunia se giraron hacia Michael— Él fue quien nos la mostró.

—Entonces no hace falta que os peleéis por esto —dijo Harry firmemente. Solo esperando que, por fin esta vez, sus padres lo escuchasen—. Si es cierto, tan solo debemos hacer venir a un profesor de Hogwarts y ver la magia por nosotros mismos, y papá tendrá que admitir que es de verdad. Y si no, entonces mamá admitirá que es falsa. En eso consiste el método experimental, de esta forma no es necesario resolver las cosas discutiendo.

El profesor se giró entonces y bajó la vista hacia él, con su habitual aire despectivo.

—Oh venga ya Harry. ¿En serio? ¿\emph{Magia}? No esperaba que \emph{precisamente tú} fueses a tomarte esto en serio, hijo, aunque solo tengas diez años. ¡La magia es simplemente lo más anticientífico que hay!

La boca de Harry se torció amargamente. Sus padres lo trataban bien, probablemente mejor que la mayoría de padres biológicos trataban a sus propios hijos. Harry había sido enviado a estudiar al mejor colegio de primaria, y cuando vieron que no estaba funcionando, contrataron profesores particulares provenientes del infinito pozo de estudiantes hambrientos. A Harry siempre lo habían animado a que estudiese todo aquello que llamara su atención, le habían comprado todos los libros que le gustasen, o le habían pagado cualquier competición de matemáticas o científica en la que se hubiese inscrito. Se le daba todo aquello razonable que pidiese, exceptuando tal vez, el mínimo atibo de respeto. Un catedrático en bioquímica de la Universidad de Oxford era imposible que fuera a escuchar los consejos de un niño. Lo escucharía para \emph{mostrar interés}, claro; es lo que todo \emph{buen padre} haría, y por tanto, como eres un \emph{buen padre}, lo haces. Pero, ¿tomar \emph{en serio} a un crío de diez años? Imposible.

A veces, Harry quería gritarle a su padre.

—Mamá —dijo Harry—. Si quieres ganarle esta discusión a papá, mira el capítulo dos del primer libro de las Conferencias Feynman sobre Física. Tiene una cita sobre cómo los filósofos hablan largo y tendido sobre los requisitos absolutos de la ciencia, y que está todo mal, ya que la única regla de la ciencia es que el árbitro último es la observación, que uno tan solo debe mirar al mundo y verbalizar lo que ve. Em... ahora mismo no me viene a la cabeza dónde encontrar información sobre una de las máximas de la ciencia, que es resolver las cosas mediante la experimentación en lugar de las discusiones—

His mother looked down at him and smiled. “Thank you, Harry. But—” her head rose back up to stare at her husband. “I don’t want to win an argument with your father. I want my husband to, to listen to his wife who loves him, and trust her just this once—”

Su madre bajó la vista sonriendo.

—Gracias, Harry. Pero —dijo mientras volvía a levantar la cabeza hacia su marido—, no intento ganarle a tu padre. Me gustaría que mi marido escuchase a su esposa, que lo quiere, y que por una vez confiase en ella y—

Harry cerró los ojos un instante. \emph{Desesperante}. Simplemente, no había esperanza para sus padres.

Sus padres habían entrado otra vez en una de \emph{esas} peleas suyas, donde su madre intentaba hacer sentir culpable a su padre, y este intentaba hacerla sentir estupida a ella.

—Me voy a mi cuarto —anunció Harry. Su voz se volvió temblorosa—. ¿Podríais intentar no discutir demasiado sobre esto? Mamá, papá, pronto sabremos lo que pasa en cualquier caso, ¿no?

—Por supuesto, Harry —dijo su padre, y ambos le dieron un beso para tranquilizarlo. Después continuaron discutiendo mientras Harry subía las escaleras hacia su habitación.

Tras cerrar la puerta de su habitación al entrar, se puso a pensar.

Realmente, él \emph{debería} estar de acuerdo con su padre. Nunca nadie ha visto pruebas de la existencia de magia, y de acuerdo con su madre, había aparentemente todo un mundo mágico ahí furea. ¿Cómo se puede mantener algo así en secreto? ¿Más magia? Eso sonaba muy sospechoso como excusa.

Obviamente su madre tenía que estar de broma, mintiendo, o se había vuelto completamente loca, cada cual peor que la anterior. Si su madre hubiese mandado la carta ella misma, eso explicaría cómo había conseguido llegar al buzón sin necesitar un sello. Un ataque de locura era, por mucho, más razonable que creer que el universo realmente era así.

El problema era que una parte de Harry estaba realmente convencida que la magia era real, y todo había comenzado en el instante en el que había visto llegar la carta del Colegio Hogwarts, de Magia y Hechicería.

Harry se frotó la frente haciendo una mueca. \emph{No creas todo en lo que piensas} decía uno de sus libros.

Pero esta extraña certeza\dots Harry se encontraba simplemente \emph{esperando} que, en efecto, un profesor de Hogwarts hiciese su aparición agitando una varita y que la magia brotase de ella. Esa extraña certeza no se esforzaba por intentar se falsificada, no inventaba excusas de antemano para explicar por qué no habría tal profesor, o que este solo sabría doblar cucharas.

\emph{Where do you come from, strange little prediction?} Harry directed the thought at his brain. \emph{Why do I believe what I believe?}
\emph{De dónde vienes, predicción mía.} Harry dirigió el pensamiento hacia su cerebro. \emph{Por qué creo en lo que creo.}

Normalmente Harry era sumamente capaz de responder dicha pregunta, pero por algún motivo, aquí se sentía completamente.

Harry se congió de hombros mentalmente. Una placa de metal en una puerta solo puede empujarse, de un tirador solo se puede tirar, y lo que se debe hacer con una hipótesis comprobable es ir y comprobarla.

He took a piece of lined paper from his desk, and started writing.
Sacó un folio de su escritorio y empezó a escribir.

\begin{writtenNote}
	\letterAddress{Estimada subdirectora}
\end{writtenNote}

Harry se detuvo reflexionando. Descartó el papel y tomó otro, sacando un milímetro más de grafito de su portaminas. Una situación como esta requería caligrafía cuidadosa.

\begin{writtenNote}
	\letterAddress{Estimada subdirectora Minerva McGonagall,}

	\letterAddress{o a quien corresponda:}

	Recientemente he recibido de su parte una carta de admisión en Hogwarts dirigida al Sr. H. Potter. Puede que no sean conscientes de que mis padres biológicos, James Potter y Lily Potter (antes llamada Lily Evans), fallecieron. Fui adoptado por la hermana de Lily, Petunia Evans-Verres, y su marido, Michael Verres-Evans.

	Es de mi total interés acudir a Hogwarts, partiendo de la base de que tal lugar exista realmente. Tan solo mi madre Petunia dice conocer la magia, pero no es capaz de utilizarla ella misma. Mi padre es una persona altamente escéptica. Yo por mi parte, me encuentro indeciso. No sé, tampoco, dónde obtener ninguno de los libros o el material listado en su carta de admisión.

	Mi madre ha mencionado que se envió un representante de Hogwarts a Lily Potter (por entonces Lily Evans) con el objeto de demostrarle a su familia que, en efecto, la magia es real, y asumo que para ayudar a Lily a obtener su material escolar. Si se puediese hacer esto con mi familia también, sería de gran ayuda.

	\letterClosing[Atentamente,]{Harry James Potter-Evans-Verres.}
\end{writtenNote}

Harry añadió entonces su dirección, dobló la carta, y la metió en un sobre donde escribió como dirección ``Hogwarts''. Tras una mayor consideración, decidió tomar una vela y dejar caer cera en la solapa del sobre sobre la que, usando la punta de una navaja, imprimió las iniciales H.J.P.E.V\@. Si realmente iba a descender a la locura de esa forma, al menos lo haría con estilo.

Then he opened his door and went back downstairs. His father was sitting in the living-room and reading a book of higher maths to show how smart he was; and his mother was in the kitchen preparing one of his father’s favourite meals to show how loving she was. It didn’t look like they were talking to one another at all. As scary as arguments could be, \emph{not arguing} was somehow much worse.

“Mum,” Harry said into the unnerving silence, “I’m going to test the hypothesis. According to your theory, how do I send an owl to Hogwarts?”

His mother turned from the kitchen sink to stare at him, looking shocked. “I—I don’t know, I think you just have to own a magic owl.”

That should’ve sounded highly suspicious, \emph{oh, so there’s no way to test your theory then}, but the peculiar certainty in Harry seemed willing to stick its neck out even further.

“Well, the letter got here somehow,” Harry said, “so I’ll just wave it around outside and call ‘letter for Hogwarts!’ and see if an owl picks it up. Dad, do you want to come and watch?”

His father shook his head minutely and kept on reading. \emph{Of course,} Harry thought to himself. Magic was a disgraceful thing that only stupid people believed in; if his father went so far as to \emph{test} the hypothesis, or even \emph{watch} it being tested, that would feel like \emph{associating} himself with that…

Only as Harry stumped out the back door, into the back garden, did it occur to him that if an owl \emph{did} come down and snatch the letter, he was going to have some trouble telling Dad about it.

\emph{But—well—that can’t \emph{really} happen, can it? No matter what my brain seems to believe. If an owl really comes down and grabs this envelope, I’m going to have worries a lot more important than what Dad thinks.}

Harry took a deep breath, and raised the envelope into the air.

He swallowed.

Calling out \emph{Letter for Hogwarts!} while holding an envelope high in the air in the middle of your own back garden was…actually pretty embarrassing, now that he thought about it.

\emph{No. I’m better than Dad. I will use the scientific method even if it makes me feel stupid.}

“Letter—” Harry said, but it actually came out as more of a whispered croak.

Harry steeled his will, and shouted into the empty sky, “\emph{Letter for Hogwarts! Can I get an owl?}”

“Harry?” asked a bemused woman’s voice, one of the neighbours.

Harry pulled down his hand like it was on fire and hid the envelope behind his back like it was drug money. His whole face was hot with shame.

An old woman’s face peered out from above the neighbouring fence, grizzled grey hair escaping from her hairnet. Mrs~Figg, the occasional babysitter. “What are you doing, Harry?”

“Nothing,” Harry said in a strangled voice. “Just—testing a really silly theory—”

“Did you get your acceptance letter from Hogwarts?”

Harry froze in place.

“Yes,” Harry’s lips said a little while later. “I got a letter from Hogwarts. They say they want my owl by the 31st of July, but—”

“But you don’t \emph{have} an owl. Poor dear! I can’t imagine \emph{what} someone must have been thinking, sending you just the standard letter.”

A wrinkled arm stretched out over the fence, and opened an expectant hand. Hardly even thinking at this point, Harry gave over his envelope.

“Just leave it to me, dear,” said Mrs~Figg, “and in a jiffy or two I’ll have someone over.”

And her face disappeared from over the fence.

There was a long silence in the garden.

Then a boy’s voice said, calmly and quietly, “What.”
